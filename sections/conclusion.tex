\cleardoublepage
\section{Future Work}

An an immediately adjacent task would be to design a similarly hands-off way to deploy the built artifacts. This would include bootstrapping infrastructure, performing \acrshort{dsu}, and runtime inspection.

Part of the recent surge of interest in the Erlang/\acrshort{otp} ecosystem is driven by alternative \acrshort{beam} languages, notably Elixir. While the described tool supports Elixir, it should be improved for better handling native dependencies



Similarly, configuration management.


Add code signing of built artifacts

Adapt to other BEAM Languages and SCM systems
Evaluate support on ARM CPU architectures.


investigate how

It would be interesting to collect empirical data on how often automated best-effort \acrshort{appup} generation using~\cite{rebar3appup} or other algorithms for \acrshort{dsu} succeeds or fails in a real-world project.

There is a large body of ongoing research studying update safeness properties of various languages and \acrshort{dsu} systems.

Statically verify update safeness properties, develop an update safety linter

% --
\cleardoublepage
\section{Conclusion.}

This work has shown how the Erlang/\acrshort{otp} release generation process can be automated to a degree where
