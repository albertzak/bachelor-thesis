\cleardoublepage
\section{Related Work}\label{sec:related_work}

\paragraph{Knit.}\cite{davis:knit,davis:talk} Knit aims for similar usability goals; requiring just one command to generate a \acrshort{dsu}-capable release from previous packages. Knit relies on the build tool \lstinline|rebar|, which – unlike its successor \lstinline|rebar3| - includes an algorithm similar to~\cite{rebar3appup} for generating \acrshort{appup}s. Knit optionally takes metadata hints from the developer via \emph{module attributes} to guide \acrshort{appup} generation. The project appears to be no longer maintained.

\paragraph{Distillery.}

\paragraph{Edeliver.}

\paragraph{Erlang on Nix.} The package manager Nix~\cite{nix1} borrows ideas from functional programming: immutability, pure functions, referential transparency. Its artifacts are identified via cryptographic hashes of all inputs used to build them. Nix provides repeatable environments to build packages inside, similar to what containers are used for in this work, but with stronger determinism. Nix forms the base of NixOS~\cite{nixos}, a full-featured Linux distribution where the all its components are managed by Nix via symbolic links.
Work by~\cite{erlangnix} allows dependencies of Erlang/\acrshort{otp} projects to be managed via Nix, and \cite{erlangnix2}~presents a pipeline to deploy Erlang/\acrshort{otp} projects with Nix.

% --
\cleardoublepage
\section{Future Work}

\paragraph{Hands-off deployment.}


\paragraph{Update safeness.}

Add Preprocessing step for native dependencies, asset pipeline

investigate how


Measure how often automatic appup generation succeeds/fails in a real project

Statically verify update safeness properties, develop an update safety linter
Add code signing of built artifacts

Adapt to other BEAM Languages and SCM systems
Evaluate support on ARM CPU architectures.

% --
\cleardoublepage
\section{Conclusion.}
