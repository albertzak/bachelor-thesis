\documentclass[12pt,a4paper,oneside]{book}

\usepackage[pdftex,
            pdfauthor={Albert Zak},
            pdftitle={Continuous Deployment of Erlang/OTP Applications},
            pdfsubject={Bachelor's Thesis}]{hyperref}
\usepackage[utf8]{inputenc}
\usepackage[T1]{fontenc}
\usepackage{geometry}
\usepackage{bookmark}
\usepackage{listings}
\usepackage[usenames,dvipsnames,svgnames,table]{xcolor}
\usepackage[singlespacing]{setspace}
\usepackage[acronym,nopostdot,style=super,nonumberlist,nogroupskip]{glossaries}
\usepackage{fancyhdr}

\pagestyle{fancy}
\fancyhf{}
\cfoot{\thepage}
\renewcommand{\headrulewidth}{0pt}

\makeglossaries{}
\newacronym{mfa}{MF[A]}{Module, Function, and List of Arguments}
\newacronym{bif}{BIF}{Built-in Function}
\newacronym{erts}{ERTS}{Erlang Run-Time System}
\newacronym{vm}{VM}{Virtual Machine}
\newacronym{otp}{OTP}{Open Telecom Platform}
\newacronym{sasl}{SASL}{System Architecture Support Libraries}
\newacronym{os}{OS}{Operating System}
\newacronym{vcs}{VCS}{Version Control System}
\newacronym{api}{API}{Application Programming Interface}
\newacronym{ci}{CI}{Continuous Integration}
\newacronym{cli}{CLI}{Command Line Interface}
\newacronym{ram}{RAM}{Random Access Memory}
\newacronym{dsu}{DSU}{Dynamic Software Updating}
\newacronym{tls}{TLS}{Transport Layer Security}
\newacronym{http}{HTTP}{Hypertext Transfer Protocol}
\newacronym{lxc}{LXC}{Linux Containers}
\newacronym{aws}{AWS}{Amazon Web Services}
\newacronym{beam}{BEAM}{Bogdan/Björn's Erlang Abstract Machine}
\newacronym{ecs}{ECS}{Elastic Compute Cloud Container Service}
\newacronym{cow}{CoW}{Copy-on-Write}
\newacronym{appup}{appup}{Application Upgrade Instructions}
\newacronym{relup}{relup}{Release Upgrade Instructions}


\lstset{
  language=erlang,
  numbers=left,
  numberstyle=\small\color{lightgray},
  basicstyle=\ttfamily,
  frame=tb
}

\pdfpageheight=297mm
\pdfpagewidth=210mm
\geometry{a4paper, left=30mm, right=25mm, top=30mm, bottom=30mm}

\begin{document}

\frontmatter{}

\cleardoublepage{}

\section*{Abstract}

The Erlang/OTP release process requires manual interaction at many steps. Previous work on automating Erlang/OTP Release generation has addressed parts of the process by abstracting the low-level mechanics of the build step, but tasks such as versioning and handling artifacts were left to the developer. This thesis presents a release generation tool designed for hands-off operation as part of a Continuous Integration pipeline. Tight coupling with Git allows reliable identification of releases by discarding numeric versions in favor of commit hashes, and a centralized release store takes care of handling artifacts. Evaluation of a reference implementation on six hosted Continuous Integration providers (\emph{CircleCI, Codeship, Semaphore, Shippable, Travis CI,} and \emph{Wercker}) demonstrates comparable run time and ease of setup.


\tableofcontents{}

\renewcommand{\glsnamefont}[1]{\textbf{#1}}
\doublespacing
\printglossary[nonumberlist,type=\acronymtype]
\singlespacing

\mainmatter{}

\chapter{Introduction}

% Structure borrowed from Andrej Karpathy
% http://karpathy.github.io/2016/09/07/phd/

% 1 - X (+define X if not obvious) is an important problem

Continuous Deployment of software is an important problem.
Erlang/OTP Releases has first-class support for zero-downtime hot code upgrades.

% 2 - The core challenges are this and that.

The Erlang/OTP release and deployment process requires manual interaction at many steps.
A developer needs to touch various configuration files to edit version numbers.
Existing build tools produce upgrade scripts that sometimes need manual tweaking, such as rearranging instructions because of dependency issues.
Moreover, builds created on developer's machines are not deterministic. Tooling expects previously built releases to exist in specific directories on the local file system.
Deployment is a manual multi-step process. Current best practices advocate the use of \lstinline{scp} to copy the built release to the production machines. Then one has to attach an Erlang shell to the running node and issue various commands to actually perform the upgrade. This is not an appropriate process for a modern Continuous Delivery pipeline.

% 3 - Previous work on X has addressed these with Y, but the problems with this are Z.

Previous work on automating Erlang/OTP Release generation has addressed parts of the process by abstracting the low-level mechanics of the build step, but tasks such as versioning, release management, and deployment were left to the developer.

% 4 - In this work we do W.

This thesis proposes an integrated pipeline for building and deploying Erlang/OTP projects, including: (1) An architecture of a cloud-based build system that leverages containers to guarantee deterministic builds and stores them in a central release repository; (2) a binary-diffing algorithm that generates upgrade instructions, and a hands-off way for developers to give hints to the algorithm; and (3) a node agent software that applies available upgrades to the running node.

% 5 - This has the following appealing properties and our experiments show this and that.

This has the appealing property that deployments become straightforward and stress-free. Developers deploy changes in small, focused increments  multiple times a day.

\include{related-work}

\chapter{Hot Code Loading in Erlang}
\section{Example}

\lstinputlisting[
  label={lst:loop_example},
  caption={Example of a hot-upgradable Erlang Process},
  firstline=4
]{loop_example.erl}

Listing~\ref{lst:loop_example} shows how state is passed between iterations of a server loop.

The \lstinline{spawn_link/3} function atomically spawns a process and links it to the caller to be notified when the spawned process crashes.

Its three arguments reference the function to run, including the arguments. Such triplets are commonly called \acrfull{mfa}. Note that \lstinline|?MODULE| is a macro which is expanded to the name of the current module. Here, the \lstinline{loop/1} function is given an initial state of \lstinline{0}. When the process receives a message matching the pattern of the tuple \lstinline|{increment, _}|, it calculates the new state, responds, and applies the new state to itself in a tail-recursive call.

Calling the \acrfull{bif} \lstinline|c/1| compiles and loads a new version of a module into memory. However, any processes spawned so far will continue to run the old version of the code. Note that the Erlang \acrshort{vm}'s code server keeps no more than two pointers to the executable code for each module: One for the old, and one for the current. When a new version of a module is loaded, that new code becomes "current", and the existing code becomes "old". The old code is kept in memory as long as there are processes that run it. If a third version of a module is loaded, any processes running the old code are forcibly killed.

The only way for a process to switch itself to run a new version of a module is via a \emph{fully-qualified} or \emph{external} call. By prefixing the function call with the module name, the newest loaded version of the module is executed. Calling all functions in a fully-qualified way to always run the newest code would be possible; however doing so would make it impossible to transform state between versions.

In this example, when the process receives a \lstinline{upgrade} message, it may use a fully-qualified call to the new version's upgrade function. The upgrade function transforms the current state into the format that the new version expects. In this example, the state is just passed through as-is. After the state transformation, the process switches itself over to the new module.

\section{Limitations of Hot Code Loading}

The following subsections describe concerns specific to Erlang. For a more general overview of potential pitfalls in dynamically updating software, refer to \cite{gregersen:phenomena} and \cite{hicks}.

\subsection{Processes Blocking with No Timeouts}

To perform a hot code swap of a module, any processes that run this module's code must be suspended. The \acrshort{vm} can only preempt processes at \emph{reductions}; that is when the process calls any function.\footnote{The term \emph{reduction} comes from Erlang's Prolog heritage. Reductions are roughly equal to function calls, however there are some \acrshort{bif}s that don't count as reductions.} When a process blocks while waiting for a message, it doesn't perform any reductions, thus can't be suspended, and can't upgrade to the new version of the code.

In practice, this is not a problem when implementing \acrshort{otp} behaviors such as generic servers, as they transparently handle timeouts and code change events. However, developers spoiled by \acrshort{otp} may forget that a plain \lstinline|receive| statement should have a timeout. It may be possible to statically analyze an Erlang codebase for such issues.

\subsection{Dependencies and Ordering}

Complicated or circular dependencies can make it difficult or even impossible to decide in which order things must be done without risking runtime errors during an upgrade or downgrade.\cite[352]{doc:otp} Dependencies can exist between modules, processes and nodes.

\subsection{Keeping State in Records}

Erlang developers often use \emph{Records} to keep complex state. They are nothing but syntactic sugar over tuples. When tuples get too large (eg. more than three elements), it gets confusing to pattern match against elements by index. Erlang records are a way to define names for the elements of a tuple. Using records allows to conveniently pattern match against specific fields by a name instead of their index.


\subsubsection{Modifying the Structure of Records}
Record declarations are static and can't be changed at runtime. In fact, records don't even exist at runtime. The Erlang compiler \emph{de-sugars} the record syntax into tuples. There is no language support for changing the structure of a record via a hot upgrade. If a record declaration is changed between module versions, an existing record can't be parsed by the new version of the module.

There have been attempts by the Erlang\footnote{\url{https://github.com/andytill/aversion}} and Elixir\footnote{\url{https://github.com/yrashk/exrecord}} communities to add a notion of versioned records. Another project is the \lstinline|exprecs| \emph{parse transformation} plugin. Note that the official Erlang documentation states: "Programmers are strongly advised not to engage in parse transformations. No support is offered for problems encountered." While useful, manually versioning records should be reserved for exceptionally complex cases. \cite{rebar3appup} developed a way of comparing two compiled \lstinline|*.beam| files and injecting the necessary code for converting between changed record definitions.

\subsubsection{Sharing Record Declarations}
Records should be specified local to a single module, but they can also be defined in Erlang header files (\lstinline|*.hrl|) that are shared between modules. \cite{davis:talk} recommends disallowing record definitions in Erlang header files. Instead, records that need to be accessible to multiple modules or processes should be encapsulated by their own module which provides functions to get and set fields, and transparently handles versioning.

\subsubsection{Using Maps Instead of Records}
Since OTP release 17, Erlang has \emph{maps}--a first class \emph{associative array} or \emph{dictionary} data type that allows pattern matching and is more performant than the other key-value data structures\footnote{Key-Value Dictionary as Ordered List (\lstinline|orddict|), Key-Value Dictionary (\lstinline|dict|), General Balanced Trees (\lstinline|gb_trees|)} in Erlang. Maps are designed as a replacement for records. Since fields can be added or removed at any time, maps are easy to upgrade in a code change function. However, we lose the strictness of records. \cite{ferd} notes that while a bad upgrade between records would crash immediately and loudly, a bad upgrade between maps might silently corrupt state and lead to more obscure bugs later.


\subsection{Passing Anonymous Functions as Callbacks}

Processes may pass anonymous functions to each other. For example, a worker process might accept a message that triggers a network request. To signal completion, the requester passes an anonymous function as a \emph{callback} to the worker process.

State may be implicitly passed to the callback in a \emph{closure}. This must not be done when planning to use hot code swapping. Any process that holds a reference to an anonymous function defined in an old module is terminated by the \acrshort{vm}.

\cite{davis:talk} reports that the use of anonymous functions is the most common cause of failed upgrades; and recommends to avoid passing them as long-lived callbacks between processes. Instead, the calling module should specify the callback function in fully-qualified form. The calling module can pass additional arguments as \lstinline|{M,F,A}|; and the callee must then pass through the supplied arguments to the callback, including any additional data, such as the result of the remote call.

However, as such pass-though data is of no concern to the callee, \cite{carlsson:parameterized} argued that it should not affect the implementation of the callee. He proposed to add \emph{parameterized modules} (also called \emph{"abstract modules"}) to Erlang. An experimental implementation made it possible, as in Object Oriented Programming, to define modules with free variables, call a constructor function to instantiate, and pass references to such module instances as callbacks.
However, parameterized module support was never officially documented and was removed\footnote{Commit to the Erlang/OTP Repository; \url{https://github.com/erlang/otp/commit/35adf88290339ecdbbcd0a1290032d599bda26c4}, accessed August 20, 2017} in version R16\footnote{Note that Erlang/OTP releases before 17.0 followed the version scheme \lstinline|R<major>[B<minor>]|}.

\section{Structure of Erlang/OTP Projects}

The smallest unit of organization is the \emph{module}, which is a file ending in \lstinline|*.erl| containing function definitions and module attributes. Note that modules are not namespaced in Erlang\footnote{Elixir works around this by prefixing modules with ``\lstinline|Elixir.|'' and concatenating the names of nested modules with a period.}. Modules related to each other are packaged as an \emph{OTP Application}. There are two kinds of Applications:

\begin{enumerate}
  \item \emph{Active Applications} have their own lifecycle, and include their own \emph{root supervisor} that manages the processes contained within. They are explicitly started as part of another Application. Active Applications are often referred to as just \emph{Applications}.\cite{logan:otp}

  \item \emph{Library Applications} are a collection of related modules that can be imported by other Applications. They are not meant to be started as a separate entity; they passively provide functionality.
\end{enumerate}


It is also possible to start an application as an \emph{included application}, which starts it under [another Application's] own OTP supervisor with its own strategy to restart it.\cite{ferd:anger} Note that there is no standard packaged format but plain directories, and that both Applications and Library Applications share the same directory structure.


\subsection{OTP Releases}

A \emph{Release} is a versioned set of compiled Applications and their dependencies, metadata, and lifecycle scripts. A production Erlang/OTP system usually consists of multiple Applications that may or may not depend on each other, all packaged and deployed together as a Release; usually distributed as a \lstinline|*.tar.gz| archive. The applications that make up a Release are started on one Erlang \emph{Node}, which is an \acrshort{os}-level process running the Release on top of the Erlang \acrshort{vm}. A system that is transferred to and installed at another site is called a \emph{target system}.\cite[347]{d]oc:otp}

A Release is described by a \emph{Release Resource File} \lstinline|*.rel|, which specifies the name and a version of the Release, pins the \acrshort{erts} to a specific version and lists the included applications at specific versions. The Release Resource File is compiled to \lstinline|*.script| and \lstinline|*.boot| files used by the \emph{release handler} process to start the system.


\subsubsection{System Architecture Support Libraries}

The Erlang/\acrshort{otp} distribution includes the \acrfull{sasl} application, which provides low-level services to help generate, package, and install \acrshort{otp} Releases. Specifically, the \lstinline|systools| module writes boot scripts, release upgrade files (\lstinline|relup|), and release packages (\lstinline|*.tar.gz|); while the \lstinline|release_handler| process is responsible for unpacking and installing a Release onto a running system. Consequently, Erlang/OTP systems that plan to use \acrshort{otp} facilities for release handling must depend on \acrshort{sasl}.


\subsubsection{Hot Upgrades Without OTP Releases}

There are ways to perform hot code replacement without using OTP and Releases by interfacing directly with the Erlang Code Server, via the low-level \lstinline|code| module specifically using the \lstinline|code:l/1| ("load") and \lstinline|code:nl/1| ("network load") functions. Both take the name of a compiled \lstinline|*.beam| file and load the code into either the current \acrshort{vm} or into all Nodes connected via the Erlang Distribution Protocol. Doing so is fine in development, however for production deployments the following issues arise:

\begin{enumerate}
\item No \lstinline|code_change| methods of any \acrshort{otp} behaviors are called when using \lstinline|code:l/1| or \lstinline|code:nl/1|.
\item Modules loaded onto connected Nodes with \lstinline|code:nl/1| are wiped when the remote Node restarts.
\item It's hard to determine what code a given Node is running.
\end{enumerate}



\chapter{Proposed Design Decisions}
This chapter describes the reasoning behind the components, the general design decisions, and how they play together to enable a highly automated Continuous Integration pipeline for the Erlang/OTP release handling process.

\section{Design Goals}

The primary design goal is to require as little interaction as possible from the developer to build a release.

% TODO: Why?

First, we propose replacing traditional version numbers with commit hashes to (1) relieve the developer of handling versioning, and (2) to guarantee uniqueness and identifiability of each revision.

Second, we propose that artifacts be built inside a standardized environment using declarative container technology to guarantee repeatability of builds.

\section{Automated Versioning}

The version number of a release must uniquely identify a snapshot of the system's code and its dependencies. Up until now, developers had to manually edit version strings in various files each time they wished to create a release. The current process is error-prone and slow, leading to bigger, less focused releases that are harder to apply. The solution proposed in this section replaces numeric version strings with automatically generated cryptographic hashes of the changesets. Developers are relieved of coordinating versioning, and the system is able to provide uniqueness and identifiability guarantees.

The Erlang/OTP release handling system expects version numbers to exist in various metadata files. Multiple books recommend to use traditional numeric version strings, as they usually can be parsed by some of the existing tooling.\cite[252, 322]{logan:otp, cesarini:otp, ferd}

\subsection{Manually Incrementing Version Numbers}
Currently, developers have to remember to manually increment version numbers before committing a change. This leads to various problems where multiple revisions of the code base share the same version number:

\begin{enumerate}
  \item If a developer forgets to increment the version number, different snapshots of the code base share the same version number, thus breaking uniqueness.
  \item Multiple developers working on a single project in parallel may fail to coordinate the next increment, thus breaking uniqueness.
\end{enumerate}

\subsubsection{Storing Metadata in the Version String}
Version strings have also been used to store various other meta-information about an artefact. Categorizing changes into major and minor based on the developer's assumptions of whether or not a changeset is backwards compatible can certainly be very useful for the consumers of library modules. In 2010, GitHub introduced Semantic Versioning and encouraged \emph{all} software projects to adopt it. However, \cite{rae:semver} found that the current mechanisms to signal interface instability are not used properly.

\subsection{Numeric Versioning Schemes}
On the surface, traditional version numbers seem to have a number of advantages over hashes: They are ordered, and they might provide some information about the release at a glance. However, total order is not guaranteed taking into account the above issues.

\subsubsection{Advantages in Erlang/OTP}
In regards to Erlang/OTP systems, one advantage of numeric versioning schemes is that Erlang has built-in support for comparing version strings. For example, if multiple versions of an application are available, Erlang picks the one with the highest version number. Additionally, application upgrade files (\lstinline|appup|) can specify not only one specific version to upgrade from or downgrade to; but may use regular expressions (regex) to define a range of acceptable source versions.\cite{cesarini:otp}

\subsection{Cryptographic Versioning Schemes}
Replacing traditional numeric version numbers with the cryptographic hash of a commit guarantees uniqueness and reliably links the release with the commit. Version number schemes differ between \acrfull{scm}s. We can't however simply write a commit hash into a version-controlled file, as there is no way to know the hash of a commit before committing.

\subsubsection{Second Commit After Commit}
One rudimentary workaround could be to follow with another commit immediately that writes the hash of the previous commit to the files that need updating. However, doing so would (1) pollute the commit history, and (2) create ambiguity when attempting to checkout the revision of the code as reported by a release.

\subsubsection{Temporary Edit After Commit}
The proposed solution is to keep version numbers out of the \acrfull{scm} repository. Any version-controlled metadata files must not define any version strings identifying any version-controlled code from the same repository. The canonical version string of a revision is its cryptographic commit hash, and is generated once and only after a changeset is finalized. The proposed pipeline is designed to integrate with the Git \acrshort{scm} system. The SHA-1 hashes identifying Git commits are dynamically inserted as version strings in metadata files when building a release. All edits happen on a copy of the code, and are discarded when the build is finished, without ever being persisted in the repository. The process is detailed in the section on Deterministic Builds.

\subsubsection{Identifying Changed Sub-Applications}
Complex Erlang/OTP applications ("umbrella applications") are often split up into several sub-applications, that live in separate directories and are developed and deployed together. Another advantage of using commit hashes as version strings is that it becomes trivial to identify changed sub-applications by retrieving the last commit that changed any file inside a certain directory.

\begin{lstlisting}[
  label={lst:githash},
  caption={Git command to retrieve the hash of the last commit made in a subdirectory}
]
git rev-list -1 HEAD -- "$directory"
\end{lstlisting}


\subsubsection{Dealing With Dirty Working Trees}
There is one issue with the above "temporary edit after commit" solution; when a developer edits a checked out copy, does not commit the changes, and invokes the proposed build pipeline. Git calls this situation a "dirty working tree", and it is a normal occurrence during development. The command given in Listing \ref{lst:githash} would still return the hash of the last commit, even though the state of the files in the working directory has changed since. The build system needs to be able to detect such situations. Since the proposed build pipeline is designed to be run on a \acrshort{ci} server from a "fresh" clone of the repository, the only sensible course of action is to fail if the working tree is dirty.

The build system first needs to invoke the command given in Listing \ref{lst:gitchanges} to determine if any changes were made to tracked (i.e. known to git) files. The exit code \lstinline|0| means that there were no differences, while \lstinline|1| indicates that some tracked files were changed.

\begin{lstlisting}[
  label={lst:gitchanges},
  caption={Git command to check for changes to tracked files}
]
git diff-index --quiet HEAD --
\end{lstlisting}

Any newly created and thus untracked files are not reported by the command in Listing \ref{lst:gitchanges}. To get a list of untracked files, the build system also needs to call the command given in Listing \ref{lst:gituntracked}. Note that now the tool has to parse the output instead of just checking the return code: If there are no untracked files, the output is empty.

\begin{lstlisting}[
  label={lst:gituntracked},
  caption={Git command to print a list of all untracked (new) files}
]
git ls-files --exclude-standard --others
\end{lstlisting}

Both commands in Listings \ref{lst:gitchanges} and \ref{lst:gituntracked} are "plumbing" commands specifically recommended by the Git developers for use in scripts, as their \acrshort{api} is promised to be kept stable.\cite{man:git}

\section{Deterministic Builds}

\subsection{Building in Linux Containers}

\subsubsection{The Command Line Interface Tool}
Since all of the action happens inside containers, the \acrshort{cli} tool can be distributed as a single compact shell script. When the \acrshort{cli} tool is invoked, it acts as a thin wrapper around the Docker client. The script does little more than start a container, and set its environment variables and volume paths. Distributing a single shell script with only one dependency on the Docker client may ease adoption of the proposed tool among developers. Installing the tool on a \acrshort{ci} server can also be made trivial, as Listing \ref{lst:curlpipesh} shows.

\begin{lstlisting}[
  label={lst:curlpipesh},
  caption={CLI tool installation command}
]
curl https://get.beamup.io/install | /bin/sh
\end{lstlisting}

A single line consisting of a controversial\footnote{\url{https://curlpipesh.tumblr.com}} chain of easily copy/pasteable shell commands can download the script and execute it.


\subsection{Non-destructive Editing of Configuration Files}

As described in Section "Automated Versioning" on Page \pageref{tempeditaftercommit}, the build tool needs to make various edits to metadata and configuration files of the project. However, any editing needs to happen non-destructively--i.e. without changing the actual contents of any file in the working tree, and any files created during the build process should be removed on termination. The simplest way to guarantee non-destructiveness is to copy the working tree to a separate scratch location for modifying.

\subsubsection{Bind Mounting the Host's Working Directory}
When the build command is invoked, the \acrshort{cli} wrapper script starts a builder container and bind mounts the current working directory in read-only mode. Restricting the container to only read from the mounted volume guarantees that the host's working always tree stays intact. A bind mount is the most performant way to share files of the host with the container.\cite{docker:docs} However, there is no way to allow temporary changes from within the container which are not propagated to the host's file system, but stay local to the mounted volume inside the container. Since the build tools running inside the container expect to be able to change files, the contents from the bind-mounted working directory need to be copied to another writable location inside the container.

\subsubsection{Temporary Scratch File System}
The container is also assigned a \lstinline|tmpfs| mount, which is a container-accessible volume whose contents are stored in the host's \acrshort{ram} for as long as the container is running.\cite{docker:docs} When the container is started, the contents of the bind mount are copied to the \lstinline|tmpfs| mount where they can be freely edited by the build tools without affecting the files in the original working tree on the host. There is an inital startup performance penalty when copying the working tree to the \lstinline|tmpfs|, however later file operations on the ramdisk are faster.
% TODO: Reference to Latency Measure in Chapter XX
Listing \ref{lst:dockermount} demonstrates running the containerized build process with the current host directory bind mounted at \lstinline|/source| and a \lstinline|tmpfs| ramdisk mounted at \lstinline|/scratch|.

\begin{lstlisting}[
  label={lst:dockermount},
  caption={Mounting a host volume and a scratch disk inside a container}
]
  docker run --rm -it \
    --mount type=bind,source="$(pwd)",target=/source \
    --mount type=tmpfs,destination=/scratch \
    beamup/builder
\end{lstlisting}


\input{appups}

\include{future-work}
\cleardoublepage
\section{Future Work}

An an immediately adjacent task would be to design a similarly hands-off way to deploy the built artifacts. This would include bootstrapping infrastructure, performing \acrshort{dsu}, and runtime inspection.

Part of the recent surge of interest in the Erlang/\acrshort{otp} ecosystem is driven by alternative \acrshort{beam} languages, notably Elixir. While the described tool supports Elixir, it should be improved for better handling native dependencies



Similarly, configuration management.


Add code signing of built artifacts

Adapt to other BEAM Languages and SCM systems
Evaluate support on ARM CPU architectures.


investigate how

It would be interesting to collect empirical data on how often automated best-effort \acrshort{appup} generation using~\cite{rebar3appup} or other algorithms for \acrshort{dsu} succeeds or fails in a real-world project.

There is a large body of ongoing research studying update safeness properties of various languages and \acrshort{dsu} systems.

Statically verify update safeness properties, develop an update safety linter

% --
\cleardoublepage
\section{Conclusion.}

This work has shown how the Erlang/\acrshort{otp} release generation process can be automated to a degree where


\include{lists}

\bibliography{lit}
\bibliographystyle{alpha}

\end{document}
