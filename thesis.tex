\documentclass[12pt,a4paper,oneside]{book}

\usepackage[pdftex,
            pdfauthor={Albert Zak},
            pdftitle={Continuous Deployment of Erlang/OTP Applications},
            pdfsubject={Bachelor's Thesis}]{hyperref}
\usepackage[utf8]{inputenc}
\usepackage[T1]{fontenc}
\usepackage{geometry}
\usepackage{bookmark}
\usepackage{listings}
\usepackage[usenames,dvipsnames,svgnames,table]{xcolor}
\usepackage[singlespacing]{setspace}
\usepackage[acronym,nopostdot,style=super,nonumberlist,nogroupskip]{glossaries}
\usepackage{fancyhdr}

\pagestyle{fancy}
\fancyhf{}
\cfoot{\thepage}
\renewcommand{\headrulewidth}{0pt}

\makeglossaries{}
\newacronym{mfa}{MF[A]}{Module, Function, and List of Arguments}


\lstset{
  language=erlang,
  numbers=left,
  numberstyle=\small\color{lightgray},
  basicstyle=\ttfamily,
  frame=tb
}

\pdfpageheight=297mm
\pdfpagewidth=210mm
\geometry{a4paper, left=30mm, right=25mm, top=30mm, bottom=30mm}

\begin{document}

\frontmatter{}

\cleardoublepage{}

\chapter*{Abstract}

The Erlang/OTP Release process requires manual interaction at many steps. Previous work on automating Erlang/OTP Release generation has addressed parts of the process by abstracting the low-level mechanics of the build step, but tasks such as versioning, writing upgrade instructions, and handling artifacts were left to the developer. This thesis proposes  the design of a build tool that trades customizability for hands-off operation as part of a Continuous Integration (CI) pipeline.


\tableofcontents{}

\renewcommand{\glsnamefont}[1]{\textbf{#1}}
\doublespacing
\printglossary[nonumberlist,type=\acronymtype]
\singlespacing

\mainmatter{}

\section{Introduction}

The contribution of this work is a tool that automates the steps required to generate deployable artifacts of Erlang/\acrshort{otp} projects using a single command that runs without user interaction.
This section explains relevant terminology and shows where manual interaction is required to generate \acrshort{otp} Releases. An overview of existing tooling and how they build on top of each other follows.

\subsection{Erlang/OTP}

Erlang is a programming language for building concurrent, distributed systems with high availability requirements. Initially designed by Ericsson for telephone exchanges, it has been embraced by industries with similar needs: finance, gaming, betting, messaging, middleware, and databases. Development of Erlang took place starting in 1986 at the Ericsson Computer Science Laboratory, and in 1998 Erlang was released as Open Source.~\cite{armstrong2007history}

\acrshort{otp} stands for \acrlong{otp}, and is a combination of library applications, design patterns, conventions, and documentation. Erlang is almost always used in conjunction with \acrshort{otp}, hence the name Erlang/\acrshort{otp}.~\cite{ferd}

The language is often described as functional, although it is not strictly side-effect free or referentially transparent. Erlang compiles to bytecode, which is executed by a \acrfull{vm}. The current Erlang \acrshort{vm} – the \acrshort{beam}, short for \acrlong{beam} – is written in C and supports various machine architectures. Multiple languages that compile to \acrshort{beam} bytecode have been created, most famously \emph{Elixir} and \emph{Lisp Flavored Erlang}. Actor model processes are the language's concurrency primitives: Functions can be spawned to create lightweight \acrshort{beam} processes. They are different from \acrshort{os} processes or threads, being preemptively scheduled by the \acrshort{beam}. Erlang/\acrshort{otp} systems commonly ``run millions of processes simultaneously [with each one taking] less than a kilobyte of space.''~\cite{larson}

\paragraph{\acrshort{otp} Applications.}
Erlang code is constructed out of functions defined within \mbox{\emph{module}} files bearing an \lstinline|*.erl| extension. \emph{\acrshort{otp} Applications} group related modules into reusable units to provide well-defined start and stop semantics, including an \emph{application resource file} (\lstinline|*.app|) containing additional metadata such as a version string and a list of other applications that this application depends on, and which need to be started beforehand. Every application has a dependency on at least \lstinline|kernel| and \lstinline|stdlib|, and both must be specified in the application resource file.~\cite{doc:otp}

\acrshort{otp} also enforces a certain directory structure for applications.~\cite{logan:otp}

\paragraph{\acrshort{otp} Releases.} Whole projects consisting of multiple applications are packaged and deployed as \acrshort{otp} Releases. They are described by a release resource (\lstinline|*.rel|) file, which specifies additional metadata, such as a version string for the entire release, the included version of the \acrfull{erts}, and a list of applications with their respective version strings that comprise the release.

From this release resource file, various tools can be used to create a boot script and assemble the release into a single compressed tarball (\lstinline|*.tar.gz|) package.~\cite{doc:otp} A packaged release contains everything necessary to bootstrap an \emph{embedded target system} on another machine, also called a \emph{node}. Depending on the tool used to generate the release, it may include additional convenience scripts to upgrade or inspect the target system.

\paragraph{\acrlong{dsu}.} A core feature of Erlang is its support for \acrfull{dsu}, also referred to as on-the-fly upgrading, or hot code loading. The \acrshort{beam} keeps up to two versions of a module loaded in memory, and both versions of the code may run side by side.~\cite{cesarini:otp} \acrshort{otp} provides generic \emph{behaviours} that \emph{callback modules} can implement to normalize start, stop and upgrade semantics, among others.~\cite{doc:otp} Erlang systems constructed according to \acrshort{otp} patterns, grouped into \acrshort{otp} Applications, and packaged as \acrshort{otp} Releases enjoy additional support for \acrshort{dsu} via instruction files: \acrshort{appup}s and \acrshort{relup}s.

First, there are high-level, often hand-written \acrfull{appup} files, one for each \acrshort{otp} Application. These files are fed into release generation tools where they are translated and combined into a single low-level \acrshort{relup} file, thus making a given release \acrshort{dsu}-capable.~\cite{doc:otp} The \acrshort{relup} file contains instructions on how to upgrade a node running a previous version. A single release package can include one \acrshort{relup} file which may know how to upgrade from multiple previous releases. These files also contain instructions on how to downgrade to the previous version in case the upgrade fails.~\cite{doc:otp}

\subsection{Problem}\label{sec:problem} Most existing release generation tools require manual interaction at various steps, and are generally not trivial to set up and use out of the box in a non-interactive build environment, such as a \acrfull{ci} pipeline. Additionally, there are some pitfalls when developers assemble releases on, for example, their \emph{macOS} development machines and then attempt to start them on \emph{Linux} in production~\cite{cesarini:otp}: This fails with a non-obvious error. To generate a \acrshort{dsu}-capable \acrshort{otp} Release, a developer needs to manually write \acrshort{appup} files and increment version numbers for all changed applications. The \acrshort{otp} Release resource file has an additional, separate version number that needs to be updated between commits to be able to perform \acrshort{dsu}. Then, to generate upgrade instructions, previous releases need to be fetched and unpacked. Lastly, the developer has to invoke various commands to assemble the final release package tarball.~\cite{ferd}

This process is tedious to perform manually, and impedes frequent deployment of small changes. Likewise, complex code changes are more likely to fail when applied via \acrshort{dsu}.~\cite{hicks} The unfortunate consequence is that developers are discouraged from using Erlang's \acrshort{dsu} capabilities unless absolutely necessary.~\cite{ferd}

\subsection{Contribution}

This work contributes design, implementation and evaluation of a \acrshort{dsu}-capable release generation tool for Erlang/\acrshort{otp} projects named \emph{BeamUp}. Its aim is to run without user interaction and to require as little configuration as possible.

\paragraph{Goal.} The goal is to develop a prototype implementation that produces \acrshort{otp} release packages. The tool should be easy to setup on major hosted \acrfull{ci} platforms~\cite{dig2016usage}, and integrate with the Git \acrfull{vcs}~\cite{sink2011version} to use commit hashes as version strings, instead of requiring the developer to hand-increment version numbers.~\cite{maste2016} It must handle generating \acrshort{dsu} instructions, as well as assembling and storing \acrshort{otp} Releases in a completely automated, hands-off way so that the tool may be used as part of a \acrshort{ci} pipeline.

\paragraph{Method.} The thesis describes in detail the architectural and design decisions made while iteratively implementing said build tool. The prototype is evaluated for correctness and run time performance on six hosted \acrshort{ci} platforms that provide a free tier: \emph{CircleCI, Codeship, Semaphore, Shippable, Travis CI,} and \emph{Wercker}. Finally, the work discusses limitations and advantages of the proposed tool.

\cleardoublepage
\subsection{State of the Art}\label{sec:sota}

The proposed tool, \emph{BeamUp}, relies on several layers of existing tooling, as visualized in figure~\ref{fig:tools}. At the lowest level, Erlang/\acrshort{otp} ships with the \acrfull{sasl} that include the two basic building blocks for \acrshort{dsu} support: \lstinline|systools| provides low-level functions for \emph{offline} release generation, and \lstinline|release_handler| is used to perform an \emph{online} hot-upgrade of a running node.~\cite{doc:otp}
Elixir projects are built with \lstinline|mix| (not pictured) and releases are generated with \lstinline|distillery|~\cite{distillery}, which handles generation of upgrade instructions itself, and directly interacts with \lstinline|systools|. Erlang projects, however, need more coaxing: Modules and dependencies are compiled with \lstinline|rebar3|, \acrfull{appup} files are generated on a best-effort basis by comparing \acrshort{beam} bytecode via a plugin to \lstinline|rebar3|~\cite{rebar3appup}, while release assembly is done by another intermediary tool, \lstinline|relx|~\cite{loder2016production}.

\vspace{2cm}
\begin{figure}[h]
  \begin{tikzpicture}[sibling distance=40mm,
    level distance=14mm,edge from parent,>=latex']
    \tikzstyle{edge from parent}=[draw,<-]
    \node (edeliver) at (0.9, 5) {\lstinline|edeliver|};
    \node (beamup) at (-2, 6.5) {\lstinline|beamup|};

    \node {\lstinline|SASL|} [grow'=up] {
      child {node {\lstinline|systools|}
        child {node (relx) {\lstinline|relx|}
          child {node (rebar3) {\lstinline|rebar3|}}}
        child {node (distillery) {\lstinline|distillery|}}}
      child {node (release_handler) [text=black!40] {\lstinline|release_handler|} edge from parent[draw=black!40]}
    };

    \node (appup) [left=10mm of rebar3,text centered,text width=13mm] {\lstinline|appup| \lstinline|plugin|};

    \draw[->] (appup) -- (rebar3);

    \draw[->] (edeliver) -- (distillery);
    \draw[->] (edeliver) -- (relx);
    \draw[->,draw=black!40] (edeliver) -- (release_handler);
    \draw[->,draw=black!40] (distillery) -- (release_handler);

    \draw[->] (beamup) -- (distillery);
    \draw[->] (beamup) -- (rebar3);
    \draw[->] (beamup) -- (appup);
    \draw[->,dashed,draw=black!40] (beamup.east) .. controls (5,6) and (2.3,5) .. node[very near start,sloped,above,text=black!40] {\small{(planned)}} (release_handler.32);

  \end{tikzpicture}
  \caption{Dependencies between selected tools to create and handle releases.}\label{fig:tools}
\end{figure}

\cleardoublepage
\section{Related Work}\label{sec:related_work}

There have been several attempts to automate Erlang/\acrshort{otp} release generation.

\emph{Sinan}~\cite{sinan} was a widely used tool to simplify assembling releases, but did not automate generation of \acrshort{appup}s. The project is deprecated since 2012.

\emph{Knit}~\cite{davis:knit,davis:talk} aimed for simple operation; requiring just one command to generate a \acrshort{dsu}-capable release from previous packages. Knit relied on the build tool \lstinline|rebar|, which – unlike its successor \lstinline|rebar3| - included an algorithm similar to~\cite{rebar3appup} for generating \acrshort{appup}s. Knit optionally took metadata hints from the developer via \emph{module attributes} to guide \acrshort{appup} generation. The project was last updated in 2014 and appears to be abandoned.

\emph{Relflow}~\cite{relflow} integrates with the Git \acrshort{vcs} and automatically versions releases by rewriting the the \lstinline|rebar3| config file. It uses its own \acrshort{appup} generation algorithm that relies on Git to detect what modules have changed between versions.

\emph{Tetrapak}~\cite{tetrapak} is an alternative release assembling tool that outputs artifacts as Debian packages. It too supports configurable automated versioning from \acrshort{vcs} commits, but has no support for \acrshort{dsu}.

\emph{Edeliver}~\cite{edeliver,talk:edeliver} shares very similar goals to the proposed tool. It supports Erlang and Elixir projects, coaxing any of the following tools used on a project: \lstinline|rebar|, \lstinline|relx|, \lstinline|exrm|, or \lstinline|distillery|. It generates \acrshort{dsu}-capable releases with configurable auto-generated version strings. To provide a repeatable build environment, Edeliver opens a \acrfull{ssh} tunnel to a separate build host and assembles the release remotely. Written in \lstinline|bash|, it has no additional dependencies. The tool alo handles deployment via \acrshort{ssh}, including remotely performing \acrshort{dsu}.

A different approach to release assembling relies on \emph{Nix}, a general-purpose package manager borrowing ideas from functional programming: immutability, pure functions, and referential transparency.~\cite{nix1} Its artifacts are identified via cryptographic hashes of all inputs used to build them. Nix provides repeatable environments to build packages inside, similar to what containers are used for in this work, but with stronger determinism. Nix forms the base of \emph{NixOS}~\cite{nixos}, a full-featured Linux distribution where almost all components are controlled by Nix via symbolic links.

Previous work by~\cite{erlangnix} allows dependencies of Erlang/\acrshort{otp} projects to be managed via Nix, relying on \cite{hex2nix} to provide Nix metadata, called \emph{Expressions}, for most of the Erlang and Elixir packages hosted on \lstinline|hex.pm|, the primary package repository of the Erlang/\acrshort{otp} ecosystem. Lastly,~\cite{erlangnix2}~presents a pipeline to deploy Erlang/\acrshort{otp} projects using Nix.

\chapter{Related Work}


\chapter{Hot Code Loading in Erlang}
\section{Example}

\lstinputlisting[
  label={lst:loop_example},
  caption={Example of a hot-upgradable Erlang Process},
  firstline=4
]{loop_example.erl}

Listing~\ref{lst:loop_example} shows how state is passed between iterations of a server loop.

The \lstinline{spawn_link/3} function atomically spawns a process and links it to the caller to be notified when the spawned process crashes.

Its three arguments reference the function to run, including the arguments. Such triplets are commonly called \acrfull{mfa}. Note that \lstinline|?MODULE| is a macro which is expanded to the name of the current module. Here, the \lstinline{loop/1} function is given an initial state of \lstinline{0}. When the process receives a message matching the pattern of the tuple \lstinline|{increment, _}|, it calculates the new state, responds, and applies the new state to itself in a tail-recursive call.

Calling the \acrfull{bif} \lstinline|c/1| compiles and loads a new version of a module into memory. However, any processes spawned so far will continue to run the old version of the code. Note that the Erlang \acrshort{vm}'s code server keeps no more than two pointers to the executable code for each module: One for the old, and one for the current. When a new version of a module is loaded, that new code becomes "current", and the existing code becomes "old". The old code is kept in memory as long as there are processes that run it. If a third version of a module is loaded, any processes running the old code are forcibly killed.

The only way for a process to switch itself to run a new version of a module is via a \emph{fully-qualified} or \emph{external} call. By prefixing the function call with the module name, the newest loaded version of the module is executed. Calling all functions in a fully-qualified way to always run the newest code would be possible; however doing so would make it impossible to transform state between versions.

In this example, when the process receives a \lstinline{upgrade} message, it may use a fully-qualified call to the new version's upgrade function. The upgrade function transforms the current state into the format that the new version expects. In this example, the state is just passed through as-is. After the state transformation, the process switches itself over to the new module.

\section{Limitations of Hot Code Loading}

The following subsections describe concerns specific to Erlang. For a more general overview of potential pitfalls in dynamically updating software, refer to \cite{gregersen:phenomena} and \cite{hicks}.

\subsection{Processes Blocking with No Timeouts}

To perform a hot code swap of a module, any processes that run this module's code must be suspended. The \acrshort{vm} can only preempt processes at \emph{reductions}; that is when the process calls any function.\footnote{The term \emph{reduction} comes from Erlang's Prolog heritage. Reductions are roughly equal to function calls, however there are some \acrshort{bif}s that don't count as reductions.} When a process blocks while waiting for a message, it doesn't perform any reductions, thus can't be suspended, and can't upgrade to the new version of the code.

In practice, this is not a problem when implementing \acrshort{otp} behaviors such as generic servers, as they transparently handle timeouts and code change events. However, developers spoiled by \acrshort{otp} may forget that a plain \lstinline|receive| statement should have a timeout. It may be possible to statically analyze an Erlang codebase for such issues.

\subsection{Keeping State in Records}

Erlang developers often use \emph{Records} to keep complex state. They are nothing but syntactic sugar over tuples. When tuples get too large (eg. more than three elements), it gets confusing to pattern match against elements by index. Erlang records are a way to define names for the elements of a tuple. Using records allows to conveniently pattern match against specific fields by a name instead of their index.


\subsubsection{Modifying the Structure of Records}
Record declarations are static and can't be changed at runtime. In fact, records don't even exist at runtime. The Erlang compiler \emph{de-sugars} the record syntax into tuples. There is no language support for changing the structure of a record via a hot upgrade. If a record declaration is changed between module versions, an existing record can't be parsed by the new version of the module.

There have been attempts by the Erlang\footnote{\url{https://github.com/andytill/aversion}} and Elixir\footnote{\url{https://github.com/yrashk/exrecord}} communities to add a notion of versioned records. Another project is the \lstinline|exprecs| \emph{parse transformation} plugin. Note that the official Erlang documentation states: "Programmers are strongly advised not to engage in parse transformations. No support is offered for problems encountered." While useful, manually versioning records should be reserved for exceptionally complex cases. \cite{rebar3appup} developed a way of comparing two compiled \lstinline|*.beam| files and injecting the necessary code for converting between changed record definitions.

\subsubsection{Sharing Record Declarations}
Records should be specified local to a single module, but they can also be defined in Erlang header files (\lstinline|*.hrl|) that are shared between modules. \cite{davis:talk} recommends disallowing record definitions in Erlang header files. Instead, records that need to be accessible to multiple modules or processes should be encapsulated by their own module which provides functions to get and set fields, and transparently handles versioning.

\subsubsection{Using Maps Instead of Records}
Since OTP release 17, Erlang has \emph{maps}--a first class \emph{associative array} or \emph{dictionary} data type that allows pattern matching and is more performant than the other key-value data structures\footnote{Key-Value Dictionary as Ordered List (\lstinline|orddict|), Key-Value Dictionary (\lstinline|dict|), General Balanced Trees (\lstinline|gb_trees|)} in Erlang. Maps are designed as a replacement for records. Since fields can be added or removed at any time, maps are easy to upgrade in a code change function. However, we lose the strictness of records. \cite{ferd} notes that while a bad upgrade between records would crash immediately and loudly, a bad upgrade between maps might silently corrupt state and lead to more obscure bugs later.


\subsection{Passing Anonymous Functions as Callbacks}

Processes may pass anonymous functions to each other. For example, a worker process might accept a message that triggers a network request. To signal completion, the requester passes an anonymous function as a \emph{callback} to the worker process.

State may be implicitly passed to the callback in a \emph{closure}. This must not be done when planning to use hot code swapping. Any process that holds a reference to an anonymous function defined in an old module is terminated by the \acrshort{vm}.

\cite{davis:talk} reports that the use of anonymous functions is the most common cause of failed upgrades; and recommends to avoid passing them as long-lived callbacks between processes. Instead, the calling module should specify the callback function in fully-qualified form. The calling module can pass additional arguments as \lstinline|{M,F,A}|; and the callee must then pass through the supplied arguments to the callback, including any additional data, such as the result of the remote call.

However, as such pass-though data is of no concern to the callee, \cite{carlsson:parameterized} argued that it should not affect the implementation of the callee. He proposed to add \emph{parameterized modules} (also called \emph{"abstract modules"}) to Erlang. An experimental implementation made it possible, as in Object Oriented Programming, to define modules with free variables, call a constructor function to instantiate, and pass references to such module instances as callbacks.
However, parameterized module support was never officially documented and was removed\footnote{Commit to the Erlang/OTP Repository; \url{https://github.com/erlang/otp/commit/35adf88290339ecdbbcd0a1290032d599bda26c4}, accessed August 20, 2017} in version R16\footnote{Note that Erlang/OTP releases before 17.0 followed the version scheme \lstinline|R<major>[B<minor>]|}.

\section{OTP Releases}
An \emph{\acrshort{otp} Release} is a versioned set of compiled Applications and their dependencies, metadata, and lifecycle scripts. A Release may include the \acrfull{erts} pinned to a specific version. The applications that make up a Release are started on one Erlang \emph{Node}, which is an \acrshort{os}-level process running the Release on top of the Erlang \acrshort{vm}.

The Erlang/\acrshort{otp} distribution includes the \acrfull{sasl} application, which provides low-level services to help generate, package, and install \acrshort{otp} Releases. Specifically, the \lstinline|systools| module writes boot scripts, release upgrade files (\lstinline|relup|), and release packages (\lstinline|*.tar.gz|); while the \lstinline|release_handler| process is responsible for unpacking and installing a Release onto a running system. Consequently, Erlang/OTP systems that plan to use \acrshort{otp} facilities for hot code replacement must depend on \acrshort{sasl}.

\subsubsection{Hot Upgrades Without OTP Releases}

There are ways to perform hot code replacement without using OTP and Releases by interfacing directly with the Erlang Code Server via the low-level \lstinline|code| module using the \lstinline|code:l/1| ("load") and \lstinline|code:nl/1| ("network load") functions. Both take the name of a compiled \lstinline|*.beam| file and load the code into either the current \acrshort{vm} or into all Nodes connected via the Erlang Distribution Protocol. Doing so is fine in development, however for production deployments the following issues arise:

\begin{enumerate}
\item No \lstinline|code_change| methods of any \acrshort{otp} behaviors are called when using \lstinline|code:l/1| or \lstinline|code:nl/1|.
\item Modules loaded onto connected Nodes with \lstinline|code:nl/1| is wiped when the remote Node restarts.
\item It's hard to determine what code a given Node is running.
\end{enumerate}


\subsection{Structure of an OTP Release}



\chapter{Proposed Design Decisions}
This chapter describes the reasoning behind the components, the general design decisions, and how they play together to enable a highly automated Continuous Integration pipeline for the Erlang/OTP release handling process.

\section{Design Goals}

The primary design goal is to require as little interaction as possible from the developer to build a release.

% TODO: Why?

First, we propose replacing traditional version numbers with commit hashes to (1) relieve the developer of handling versioning, and (2) to guarantee uniqueness and identifiability of each revision.

Second, we propose that artifacts be built inside a standardized environment using declarative container technology to guarantee repeatability of builds.

\section{Automated Versioning}

The version number of a release must uniquely identify a snapshot of the system's code and its dependencies. Up until now, developers had to manually edit version strings in various files each time they wished to create a release. The current process is error-prone and slow, leading to bigger, less focused releases that are harder to apply. The solution proposed in this section replaces numeric version strings with automatically generated cryptographic hashes of the changesets. Developers are relieved of coordinating versioning, and the system is able to provide uniqueness and identifiability guarantees.

The Erlang/OTP release handling system expects version numbers to exist in various metadata files. Multiple books recommend to use traditional numeric version strings, as they usually can be parsed by some of the existing tooling.\cite[252, 322]{logan:otp, cesarini:otp, ferd}

\subsection{Manually Incrementing Version Numbers}
Currently, developers have to remember to manually increment version numbers before committing a change. This leads to various problems where multiple revisions of the code base share the same version number:

\begin{enumerate}
  \item If a developer forgets to increment the version number, different snapshots of the code base share the same version number, thus breaking uniqueness.
  \item Multiple developers working on a single project in parallel may fail to coordinate the next increment, thus breaking uniqueness.
\end{enumerate}

\subsubsection{Storing Metadata in the Version String}
Version strings have also been used to store various other meta-information about an artefact. Categorizing changes into major and minor based on the developer's assumptions of whether or not a changeset is backwards compatible can certainly be very useful for the consumers of library modules. In 2010, GitHub introduced Semantic Versioning and encouraged \emph{all} software projects to adopt it. However, \cite{rae:semver} found that the current mechanisms to signal interface instability are not used properly.

\subsection{Numeric Versioning Schemes}
On the surface, traditional version numbers seem to have a number of advantages over hashes: They are ordered, and they might provide some information about the release at a glance. However, total order is not guaranteed taking into account the above issues.

\subsubsection{Advantages in Erlang/OTP}
In regards to Erlang/OTP systems, one advantage of numeric versioning schemes is that Erlang has built-in support for comparing version strings. For example, if multiple versions of an application are available, Erlang picks the one with the highest version number. Additionally, application upgrade files (\lstinline|appup|) can specify not only one specific version to upgrade from or downgrade to; but may use regular expressions (regex) to define a range of acceptable source versions.\cite{cesarini:otp}

\subsection{Cryptographic Versioning Schemes}
Replacing traditional numeric version numbers with the cryptographic hash of a commit guarantees uniqueness and reliably links the release with the commit. Version number schemes differ between \acrfull{scm}s. We can't however simply write a commit hash into a version-controlled file, as there is no way to know the hash of a commit before committing.

\subsubsection{Second Commit After Commit}
One rudimentary workaround could be to follow with another commit immediately that writes the hash of the previous commit to the files that need updating. However, doing so would (1) pollute the commit history, and (2) create ambiguity when attempting to checkout the revision of the code as reported by a release.

\subsubsection{Temporary Edit After Commit}
The proposed solution is to keep version numbers out of the \acrfull{scm} repository. Any version-controlled metadata files must not define any version strings identifying any version-controlled code from the same repository. The canonical version string of a revision is its cryptographic commit hash, and is generated once and only after a changeset is finalized. The proposed pipeline is designed to integrate with the Git \acrshort{scm} system. The SHA-1 hashes identifying Git commits are dynamically inserted as version strings in metadata files when building a release. All edits happen on a copy of the code, and are discarded when the build is finished, without ever being persisted in the repository. The process is detailed in the section on Deterministic Builds.

\subsubsection{Identifying Changed Sub-Applications}
Complex Erlang/OTP applications ("umbrella applications") are often split up into several sub-applications, that live in separate directories and are developed and deployed together. Another advantage of using commit hashes as version strings is that it becomes trivial to identify changed sub-applications by retrieving the last commit that changed any file inside a certain directory.

\begin{lstlisting}[
  label={lst:githash},
  caption={Git command to retrieve the hash of the last commit made in a subdirectory of a repository}
]
git rev-list -1 HEAD -- "$directory"
\end{lstlisting}


\subsubsection{Dealing With Dirty Working Trees}
There is one issue with the above "temporary edit after commit" solution; when a developer edits a checked out copy, does not commit the changes, and invokes the proposed build pipeline. Git calls this situation a "dirty working tree", and it is a normal occurrence during development. The command given in Listing \ref{lst:githash} would still return the hash of the last commit, even though the state of the files in the working directory has changed since. The build system needs to be able to detect such situations.

The build system first needs to invoke the command given in Listing \ref{lst:gitchanges} to determine if any changes were made to tracked (i.e. known to git) files. The exit code \lstinline|0| means that there were no differences, while \lstinline|1| indicates that some tracked files were changed.

\begin{lstlisting}[
  label={lst:gitchanges},
  caption={Git command to check for changes to tracked files}
]
git diff-index --quiet HEAD --
\end{lstlisting}

Any newly created and thus untracked files are not reported by the command in Listing \ref{lst:gitchanges}. To get a list of untracked files, the build system also needs to call the command given in Listing \ref{lst:gituntracked}. Note that now the tool has to parse the output instead of just checking the return code: If there are no untracked files, the output is empty.

\begin{lstlisting}[
  label={lst:gituntracked},
  caption={Git command to print a list of all untracked (new) files}
]
git ls-files --exclude-standard --others
\end{lstlisting}

Both commands in Listings \ref{lst:gitchanges} and \ref{lst:gituntracked} are "plumbing" commands specifically recommended by the Git developers for use in scripts, as their \acrshort{api} is promised to be kept stable.\cite{man:git}

\section{Deterministic Builds}

\subsection{Building in Linux Containers}

\subsubsection{The Command Line Interface Tool}
Since all of the action happens inside containers, the \acrshort{cli} tool can be distributed as a single compact shell script. When the \acrshort{cli} tool is invoked, it acts as a thin wrapper around the Docker client. The script does little more than start a container, and set its environment variables and volume paths. Distributing a single shell script with only one dependency on the Docker client may ease adoption of the proposed tool among developers. Installing the tool on a \acrshort{ci} server can also be made trivial, as Listing \ref{lst:curlpipesh} shows.

\begin{lstlisting}[
  label={lst:curlpipesh},
  caption={CLI tool installation command}
]
curl https://get.beamup.io/install | /bin/sh
\end{lstlisting}

A single line consisting of a controversial\footnote{\url{https://curlpipesh.tumblr.com}} chain of easily copy/pasteable shell commands can download the script and execute it.


\section{Generating Application Upgrades}


\chapter{Future Work}

\cleardoublepage
\section{Future Work}

An an immediately adjacent task would be to design a similarly hands-off way to deploy the built artifacts. This would include bootstrapping infrastructure, performing \acrshort{dsu}, and runtime inspection.

Part of the recent surge of interest in the Erlang/\acrshort{otp} ecosystem is driven by alternative \acrshort{beam} languages, notably Elixir. While the described tool supports Elixir, it should be improved for better handling native dependencies



Similarly, configuration management.


Add code signing of built artifacts

Adapt to other BEAM Languages and SCM systems
Evaluate support on ARM CPU architectures.


investigate how

It would be interesting to collect empirical data on how often automated best-effort \acrshort{appup} generation using~\cite{rebar3appup} or other algorithms for \acrshort{dsu} succeeds or fails in a real-world project.

There is a large body of ongoing research studying update safeness properties of various languages and \acrshort{dsu} systems.

Statically verify update safeness properties, develop an update safety linter

% --
\cleardoublepage
\section{Conclusion.}

This work has shown how the Erlang/\acrshort{otp} release generation process can be automated to a degree where


\renewcommand{\listtablename}{Tables}

\listoffigures
\begingroup
\let\clearpage\relax
\listoftables
\endgroup
\begingroup
\let\clearpage\relax
\lstlistoflistings
\endgroup


\bibliography{lit}
\bibliographystyle{alpha}

\end{document}
